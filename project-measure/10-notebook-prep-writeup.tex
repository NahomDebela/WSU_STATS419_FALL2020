\documentclass[]{article}
\usepackage[left=1in,top=1in,right=1in,bottom=1in]{geometry}


%%%% more monte %%%%
% thispagestyle{empty}
% https://stackoverflow.com/questions/2166557/how-to-hide-the-page-number-in-latex-on-first-page-of-a-chapter
\usepackage{color}
% \usepackage[table]{xcolor} % are they using color?

% \definecolor{WSU.crimson}{HTML}{981e32}
% \definecolor{WSU.gray}{HTML}{5e6a71}

% \definecolor{shadecolor}{RGB}{248,248,248}
\definecolor{WSU.crimson}{RGB}{152,30,50} % use http://colors.mshaffer.com to convert from 981e32
\definecolor{WSU.gray}{RGB}{94,106,113}

%%%%%%%%%%%%%%%%%%%%%%%%%%%%

\newcommand*{\authorfont}{\fontfamily{phv}\selectfont}
\usepackage{lmodern}


  \usepackage[T1]{fontenc}
  \usepackage[utf8]{inputenc}




\usepackage{abstract}
\renewcommand{\abstractname}{}    % clear the title
\renewcommand{\absnamepos}{empty} % originally center

\renewenvironment{abstract}
 {{%
    \setlength{\leftmargin}{0mm}
    \setlength{\rightmargin}{\leftmargin}%
  }%
  \relax}
 {\endlist}

\makeatletter
\def\@maketitle{%
  \pagestyle{empty}
  \newpage
%  \null
%  \vskip 2em%
%  \begin{center}%
  \let \footnote \thanks
    {\fontsize{18}{20}\selectfont\raggedright  \setlength{\parindent}{0pt} \@title \par}%
}
%\fi
\makeatother









\title{\textbf{\textcolor{WSU.crimson}{Notebook PREP for Project 01}}  }
 

%  

% \author{ \Large  \hfill \normalsize \emph{} }
\author{}


\date{}
\setcounter{secnumdepth}{}

\usepackage{titlesec}
% See the link above: KOMA classes are not compatible with titlesec any more. Sorry.
% https://github.com/jbezos/titlesec/issues/11
\titleformat*{\section}{\bfseries}
\titleformat*{\subsection}{\bfseries\itshape}
\titleformat*{\subsubsection}{\itshape}
\titleformat*{\paragraph}{\itshape}
\titleformat*{\subparagraph}{\itshape}

% https://code.usgs.gov/usgs/norock/irvine_k/ip-092225/


%\titleformat*{\section}{\normalsize\bfseries}
%\titleformat*{\subsection}{\normalsize\itshape}
%\titleformat*{\subsubsection}{\normalsize\itshape}
%\titleformat*{\paragraph}{\normalsize\itshape}
%\titleformat*{\subparagraph}{\normalsize\itshape}

% https://tex.stackexchange.com/questions/233866/one-column-multicol-environment#233904
\usepackage{environ}
\NewEnviron{auxmulticols}[1]{%
  \ifnum#1<2\relax% Fewer than 2 columns
    %\vspace{-\baselineskip}% Possible vertical correction
    \BODY
  \else% More than 1 column
    \begin{multicols}{#1}
      \BODY
    \end{multicols}%
  \fi
}





\usepackage{natbib}
\setcitestyle{aysep={}} %% no year, comma just year
% \usepackage[numbers]{natbib}
\bibliographystyle{plainnat}



\usepackage[strings]{underscore} % protect underscores in most circumstances




\newtheorem{hypothesis}{Hypothesis}
\usepackage{setspace}


%%%%%%%%%%%%%%%%%%%%%%%%%%%%%%%%%%%%%%%%%%%%%%%%%%%%%
%%% MONTE ADDS %%%

\usepackage{fancyhdr} % fancy header 
\usepackage{lastpage} % last page 

\usepackage{multicol}


\usepackage{etoolbox}
\AtBeginEnvironment{quote}{\singlespacing\small}
% https://tex.stackexchange.com/questions/325695/how-to-style-blockquote


\usepackage{soul}			%% allows strike-through
\usepackage{url}			%% fixes underscores in urls
\usepackage{csquotes}		%% allows \textquote in references
\usepackage{rotating}		%% allows table and box rotation
\usepackage{caption}		%% customize caption information
\usepackage{booktabs}		%% enhance table/tabular environment
\usepackage{tabularx}		%% width attributes updates tabular
\usepackage{enumerate}		%% special item environment
\usepackage{enumitem}		%% special item environment

\usepackage{lineno}		%% allows linenumbers for editing using \linenumbers
\usepackage{hanging}


\usepackage{mathtools}  	%% also loads amsmath
\usepackage{bm}		%% bold-math
\usepackage{scalerel}	%% scale one element (make one beta bigger font)

\newcommand{\gFrac}[2]{ \genfrac{}{}{0pt}{1}{{#1}}{#2} }

\newcommand{\betaSH}[3]{  \gFrac{\text{\tiny #1}}{{\text{\tiny #2}}}\hat{\beta}_{\text{#3}}   }
\newcommand{\betaSB}[3]{              ^{\text{#1}} _{\text{#2}} \bm{\beta} _{\text{#3}}                   }  %% bold
\newcommand{\bigEQ}{  \scaleobj{1.5}{{\ }= } }
\newcommand{\bigP}[1]{  \scaleobj{1.5}{#1 } }





\usepackage{endnotes}  % he already does this ...
\renewcommand{\enotesize}{\normalsize}
% https://tex.stackexchange.com/questions/99984/endnotes-do-not-be-superscript-and-add-a-space
\renewcommand\makeenmark{\textsuperscript{[\theenmark]}} % in brackets %
% https://tex.stackexchange.com/questions/31574/how-to-control-the-indent-in-endnotes
\patchcmd{\enoteformat}{1.8em}{0pt}{}{}

\patchcmd{\theendnotes}
  {\makeatletter}
  {\makeatletter\renewcommand\makeenmark{\textbf{[\theenmark]} }}
  {}{}



% https://tex.stackexchange.com/questions/141906/configuring-footnote-position-and-spacing

\addtolength{\footnotesep}{5mm} % change to 1mm

\renewcommand{\thefootnote}{\textbf{\arabic{footnote}}}
\let\footnote=\endnote
%\renewcommand*{\theendnote}{\alph{endnote}}
%\renewcommand{\theendnote}{\textbf{\arabic{endnote}}}


\renewcommand*{\notesname}{}

\makeatletter
\def\enoteheading{\section*{\notesname
  \@mkboth{\MakeUppercase{\notesname}}{\MakeUppercase{\notesname}}}%
  \mbox{}\par\vskip-2.3\baselineskip\noindent\rule{.5\textwidth}{0.4pt}\par\vskip\baselineskip}
\makeatother


\renewcommand*{\contentsname}{}

\renewcommand*{\refname}{}


%\usepackage{subfigure}
\usepackage{subcaption}

\captionsetup{labelfont=bf}  % Make Table / Figure bold

%%% you could add elements here ... monte says .... %%%
%\usepackage{mypackageForCapitalH}


%%%%%%%%%%%%%%%%%%%%%%%%%%%%%%%%%%%%%%%%%%%%%%%%%%%%%

% set default figure placement to htbp
\makeatletter
\def\fps@figure{htbp}
\makeatother


% move the hyperref stuff down here, after header-includes, to allow for - \usepackage{hyperref}

\makeatletter
\@ifpackageloaded{hyperref}{}{%
\ifxetex
  \PassOptionsToPackage{hyphens}{url}\usepackage[setpagesize=false, % page size defined by xetex
              unicode=false, % unicode breaks when used with xetex
              xetex]{hyperref}
\else
  \PassOptionsToPackage{hyphens}{url}\usepackage[draft,unicode=true]{hyperref}
\fi
}

\@ifpackageloaded{color}{
    \PassOptionsToPackage{usenames,dvipsnames}{color}
}{%
    \usepackage[usenames,dvipsnames]{color}
}
\makeatother
\hypersetup{breaklinks=true,
            bookmarks=true,
            pdfauthor={},
             pdfkeywords = {},  
            pdftitle={Notebook PREP for Project 01},
            colorlinks=true,
            citecolor=blue,
            urlcolor=blue,
            linkcolor=magenta,
            pdfborder={0 0 0}}
\urlstyle{same}  % don't use monospace font for urls

% Add an option for endnotes. -----

%
% add tightlist ----------
\providecommand{\tightlist}{%
\setlength{\itemsep}{0pt}\setlength{\parskip}{0pt}}

% add some other packages ----------

% \usepackage{multicol}
% This should regulate where figures float
% See: https://tex.stackexchange.com/questions/2275/keeping-tables-figures-close-to-where-they-are-mentioned
\usepackage[section]{placeins}



\pagestyle{fancy}   
\lhead{\textcolor{WSU.crimson}{\textbf{ Notebook PREP for Project 01 }}}
\chead{}
\rhead{\textcolor{WSU.gray}{\textbf{  Page\ \thepage\ of\ \protect\pageref{LastPage} }}}
\lfoot{}
\cfoot{}
\rfoot{}


\begin{document}
	
% \pagenumbering{arabic}% resets `page` counter to 1 
%    

% \maketitle

{% \usefont{T1}{pnc}{m}{n}
\setlength{\parindent}{0pt}
\thispagestyle{plain}
{\fontsize{18}{20}\selectfont\raggedright 
\maketitle  % title \par  

}

{
   \vskip 13.5pt\relax \normalsize\fontsize{11}{12} 
   
 

}

}






\vskip -8.5pt



 % removetitleabstract

\noindent  

\hypertarget{project-1-measure}{%
\section{Project 1: Measure}\label{project-1-measure}}

\hypertarget{secret-practice}{%
\paragraph{SECRET practice}\label{secret-practice}}

We will build a work product where the data stays in a SECRET or private
format. It should not be uploaded to GitHub.

\hypertarget{data-cleansing}{%
\paragraph{Data cleansing}\label{data-cleansing}}

I have provided the data in a compiled format. In the notebook
\texttt{unit\_02\_confirmatory\_data\_analysis\textbackslash{}nascent\textbackslash{}2020-10-23\_descriptive-statistics.Rmd},
I have provided some clues on how to cleanse the data. That task is
yours.

I consider changing all of the results to one unit system part of data
cleansing. You can choose ``inches'' (in) or ``centimeters'' (cm) for
your analysis depending on your culture and comfort with a given system.
This means, all of the data needs to be converted. Please recall the
\texttt{distance} work we have done, there is a library
\texttt{measurements} and a function \texttt{conv\_unit}.

\hypertarget{data-collapsing}{%
\paragraph{Data collapsing}\label{data-collapsing}}

Some people have data for a person's ``left'' and ``right'' side of the
body. I have prepared code for you to collapse that data so (we assume)
each side of the body is equal. This is one option. You can choose to
keep the overall data and address NA's (missing values) if your research
question is tied to body symmetry.

In the notebook on correlation in week 6, the section
\texttt{1.2.3.3\ Measure} has some code on to ``getOne'' measurement
from the left or right. If one is NA, it returns the other. If they both
are available, it returns the mean.

\hypertarget{data-creation}{%
\paragraph{Data creation}\label{data-creation}}

There may be a few data features you may want to create. I have the
``arm span'' and information about the ``armpits'' which would enable
you to compute the internal ``chest width'' (from armpit to armpit).
There may be other data you can create in a similar fashion.

\hypertarget{data-proportions}{%
\paragraph{Data proportions}\label{data-proportions}}

It is very likely that for each measure row, you would want to create
``scaled variables to that person's height'', also known as a
proportion.

Alternatively, you could scale everything to a person's head height.

Alternatively, you could review lots of different proportions. I
suggested at one point that the foot-size and the ``upper arm''
(elbow-pit to arm-pit) are the same size (some basic Pythagorean theorem
could get you there or close).

There are lots of possibilities, all depends on your interests.

\begin{itemize}
\item
  Some say the unit of length of a ``one foot'' that we now decompose
  into 12 inches was a function of the actual length of the King's foot
  in England, and would change when a new King was crowned.
\item
  Another measure of length, the ``cubit'' is derived from the Latin
  word for ``elbow''
\item
  Galileo Galilei, the famous Italian polymath, literally sold his body
  parts when he died (quite the entrepreneur). He had extremely long
  fingers. In the museum in Firenze, they have on display a few of the
  fingers recovered. Yes, I have seen them
  \url{https://www.museogalileo.it/it/}. Most people miss this museum
  because they are too busy admiring David's proportions at the nearby
  Academia Gallery
  \url{https://en.wikipedia.org/wiki/David_(Michelangelo)}.
\end{itemize}

\hypertarget{data-selection}{%
\paragraph{Data selection}\label{data-selection}}

Which columns are you going to use in your analysis. The ``covariates''
will be necessary to describe the sample procedure, but for your
research question maybe you just use a few of those, or none of those.

The summary statistics on the sample ``covariates'' and on the overall
data are dependent on which columns you want to research. This depends,
or is constrained by your research question. For example, in the
``Joireman paper'' we did collect a lot of other data, and even showed
them a variant of an exercise motivator (Nike Ad):

\url{http://www.mshaffer.com/arizona/videos/exercise/010.mp3}

\url{http://www.mshaffer.com/arizona/videos/exercise/101.mp4}

\url{http://www.mshaffer.com/arizona/videos/exercise/110.mp4}

\url{http://www.mshaffer.com/arizona/videos/exercise/111.mp4}

\hypertarget{project-research-question}{%
\subsection{Project Research Question}\label{project-research-question}}

Have you formulated one primary research question and possibly 2-3
subquestions. Or maybe 2-3 primary questions?

The project was initially intended to be an exploration of the original
data (distances), the computed proportions (as a function of head
height), and its relationship to correlations. However, you now have
some experience with basic clustering techniques, so you could try to
use them as well. I would say \textbf{focus} on the key research
question and don't deviate too far afield into clustering techniques
that you don't report on ``exploratory findings'' that inform your
research question.

\hypertarget{preparation-for-final-writeup}{%
\section{Preparation for Final
Writeup}\label{preparation-for-final-writeup}}

Use this space to include and run code that gets your data and research
question prepared.

Alternatively, you can use another notebook. You will submit a final
``ZIP'' file that contains the supporting documents and the final
``PDF'' product. I will build the template for the ``PDF'' product and
help with the tasks new to you in that regard.

The data cleanup, the research question, and the analysis is up to you.

I will be supporting the process of turning the results and data you
prepare into a ``work product''.




%% appendices go here!


\newpage
\theendnotes

%%%%%%%%%%%%%%%%%%%%%%%%%%%%%%%%%%%  biblio %%%%%%%%
\newpage
\begin{auxmulticols}{}
\singlespacing 
%%%%%%%%%%%%%%%%%%%%%%%%%%%%%%%%%%%  biblio %%%%%%%%
\end{auxmulticols}

\newpage
{
\hypersetup{linkcolor=black}
\setcounter{tocdepth}{3}
\tableofcontents
}



\end{document}